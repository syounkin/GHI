\documentclass[10pt,letterpaper]{article}\usepackage[]{graphicx}\usepackage[]{color}
%% maxwidth is the original width if it is less than linewidth
%% otherwise use linewidth (to make sure the graphics do not exceed the margin)
\makeatletter
\def\maxwidth{ %
  \ifdim\Gin@nat@width>\linewidth
    \linewidth
  \else
    \Gin@nat@width
  \fi
}
\makeatother

\definecolor{fgcolor}{rgb}{0.345, 0.345, 0.345}
\newcommand{\hlnum}[1]{\textcolor[rgb]{0.686,0.059,0.569}{#1}}%
\newcommand{\hlstr}[1]{\textcolor[rgb]{0.192,0.494,0.8}{#1}}%
\newcommand{\hlcom}[1]{\textcolor[rgb]{0.678,0.584,0.686}{\textit{#1}}}%
\newcommand{\hlopt}[1]{\textcolor[rgb]{0,0,0}{#1}}%
\newcommand{\hlstd}[1]{\textcolor[rgb]{0.345,0.345,0.345}{#1}}%
\newcommand{\hlkwa}[1]{\textcolor[rgb]{0.161,0.373,0.58}{\textbf{#1}}}%
\newcommand{\hlkwb}[1]{\textcolor[rgb]{0.69,0.353,0.396}{#1}}%
\newcommand{\hlkwc}[1]{\textcolor[rgb]{0.333,0.667,0.333}{#1}}%
\newcommand{\hlkwd}[1]{\textcolor[rgb]{0.737,0.353,0.396}{\textbf{#1}}}%
\let\hlipl\hlkwb

\usepackage{framed}
\makeatletter
\newenvironment{kframe}{%
 \def\at@end@of@kframe{}%
 \ifinner\ifhmode%
  \def\at@end@of@kframe{\end{minipage}}%
  \begin{minipage}{\columnwidth}%
 \fi\fi%
 \def\FrameCommand##1{\hskip\@totalleftmargin \hskip-\fboxsep
 \colorbox{shadecolor}{##1}\hskip-\fboxsep
     % There is no \\@totalrightmargin, so:
     \hskip-\linewidth \hskip-\@totalleftmargin \hskip\columnwidth}%
 \MakeFramed {\advance\hsize-\width
   \@totalleftmargin\z@ \linewidth\hsize
   \@setminipage}}%
 {\par\unskip\endMakeFramed%
 \at@end@of@kframe}
\makeatother

\definecolor{shadecolor}{rgb}{.97, .97, .97}
\definecolor{messagecolor}{rgb}{0, 0, 0}
\definecolor{warningcolor}{rgb}{1, 0, 1}
\definecolor{errorcolor}{rgb}{1, 0, 0}
\newenvironment{knitrout}{}{} % an empty environment to be redefined in TeX

\usepackage{alltt}
\usepackage[top=0.85in,left=2.75in,footskip=0.75in]{geometry}
% amsmath and amssymb packages, useful for mathematical formulas and symbols
\usepackage{amsmath,amssymb}
% Use adjustwidth environment to exceed column width (see example table in text)
\usepackage{changepage}
% Use Unicode characters when possible
\usepackage[utf8x]{inputenc}
% textcomp package and marvosym package for additional characters
\usepackage{textcomp,marvosym}
% cite package, to clean up citations in the main text. Do not remove.
\usepackage{cite}
% Use nameref to cite supporting information files (see Supporting Information section for more info)
\usepackage{nameref,hyperref}
% line numbers
\usepackage[right]{lineno}
% ligatures disabled
\usepackage{microtype}
\DisableLigatures[f]{encoding = *, family = * }
% color can be used to apply background shading to table cells only
\usepackage[table]{xcolor}
% array package and thick rules for tables
\usepackage{array}
% create "+" rule type for thick vertical lines
\newcolumntype{+}{!{\vrule width 2pt}}
% create \thickcline for thick horizontal lines of variable length
\newlength\savedwidth
\newcommand\thickcline[1]{%
  \noalign{\global\savedwidth\arrayrulewidth\global\arrayrulewidth 2pt}%
  \cline{#1}%
  \noalign{\vskip\arrayrulewidth}%
  \noalign{\global\arrayrulewidth\savedwidth}%
}
% \thickhline command for thick horizontal lines that span the table
\newcommand\thickhline{\noalign{\global\savedwidth\arrayrulewidth\global\arrayrulewidth 2pt}%
\hline
\noalign{\global\arrayrulewidth\savedwidth}}

% Remove comment for double spacing
\usepackage{setspace}
\doublespacing

% Text layout
\raggedright
\setlength{\parindent}{0.5cm}
\textwidth 5.25in
\textheight 8.75in

% Bold the 'Figure #' in the caption and separate it from the title/caption with a period
% Captions will be left justified
\usepackage[aboveskip=1pt,labelfont=bf,labelsep=period,justification=raggedright,singlelinecheck=off]{caption}
\renewcommand{\figurename}{Fig}

% Use the PLoS provided BiBTeX style
\bibliographystyle{plos2015}

% Remove brackets from numbering in List of References
\makeatletter
\renewcommand{\@biblabel}[1]{\quad#1.}
\makeatother

% Leave date blank
\date{}

% Header and Footer with logo
\usepackage{lastpage,fancyhdr,graphicx}
\usepackage{epstopdf}
\pagestyle{myheadings}
\pagestyle{fancy}
\fancyhf{}
\setlength{\headheight}{27.023pt}
\lhead{\includegraphics[width=2.0in]{PLOS-submission.eps}}
\rfoot{\thepage/\pageref{LastPage}}
\renewcommand{\footrule}{\hrule height 2pt \vspace{2mm}}
\fancyheadoffset[L]{2.25in}
\fancyfootoffset[L]{2.25in}
\lfoot{\sf PLOS}

%% Include all macros below

\newcommand{\lorem}{{\bf LOREM}}
\newcommand{\ipsum}{{\bf IPSUM}}

% added by sgy
\usepackage{url}
\usepackage{mathtools}
\usepackage{rotating}
\newcommand{\col}[2][red]{\textcolor{#1}{#2}}
\newcommand{\bi}{\begin{itemize}}
\newcommand{\ei}{\end{itemize}}
\newcommand{\package}{\emph{ITHIM}}
\newcommand{\ithim}{\emph{ITHIM}}
\newcommand{\R}{\emph{R}}
\newcommand{\sgy}[1]{{\itshape\col{#1}}}
\newcommand{\XX}{\sgy{XX}}
\newcommand{\webpage}{\url{http://www.ithim.ghi.wisc.edu}}
\newcommand{\af}{\mathrm{PAF}}
\newcommand{\mets}{MET-hrs./week}
\newcommand{\pEm}[1]{$\mathrm{PM}_{#1}$}
\newcommand{\vmt}{$\mathrm{VMT}$}
\newcommand{\logNormal}{log-normal}
\newcommand{\RR}{\mathrm{RR}}


%% END MACROS SECTION
\IfFileExists{upquote.sty}{\usepackage{upquote}}{}
\begin{document}
\vspace*{0.2in}

% Titlemust be 250 characters or less.
\begin{flushleft}
{\Large
\textbf\newline{A Comparative Risk Assessment of Active Transportation in the United
States} % Please use "title case" (capitalize all terms in the title except conjunctions, prepositions, and articles).
}
\newline
% Insert author names, affiliations and corresponding author email (do not include titles, positions, or degrees).
\\
Samuel Younkin\textsuperscript{1},\ %,2\Yinyang},
Jason Vargo\textsuperscript{1, 2}\ %\Yinyang},
Neil Maizlish\textsuperscript{3},\ %\Yinyang},
\ldots,\
Jonathan Patz\textsuperscript{1, 2}%,3\textcurrency},
%Name4 Surname\textsuperscript{2},
%#Name5 Surname\textsuperscript{2\ddag},
%Name6 Surname\textsuperscript{2\ddag},
%Name7 Surname\textsuperscript{1,2,3*},
%with the Global Health Institute\textsuperscript{\textpilcrow}
\\
\bigskip
\textbf{1} Global Health Institute, University of Wisconsin{\textendash}Madison, Madison, WI, USA
\\
\textbf{2} Nelson Institute for Environmental Studies Center for Sustainability and the Global Environment, University of Wisconsin{\textendash}Madison, Madison, WI, USA
\\
\textbf{3} Berkeley, CA, USA
\\
%\textbf{2} Affiliation Dept/Program/Center, Institution Name, City, State, Country
%\\
%\textbf{3} Affiliation Dept/Program/Center, Institution Name, City, State, Country
%\\
\bigskip

% Insert additional author notes using the symbols described below. Insert symbol callouts after author names as necessary.
%
% Remove or comment out the author notes below if they aren't used.
%
% Primary Equal Contribution Note
%\Yinyang These authors contributed equally to this work.

% Additional Equal Contribution Note
% Also use this double-dagger symbol for special authorship notes, such as senior authorship.
% \ddag These authors also contributed equally to this work.

% Current address notes
% \textcurrency Current Address: GHI UW{\textendash}Madison, 1300
% University Ave., Madison, WI, 53706, USA
% WI, USA % change symbol to "\textcurrency a" if more than one current address note
% \textcurrency b Insert second current address
% \textcurrency c Insert third current address

% Deceased author note
%\dag Deceased

% Group/Consortium Author Note
%\textpilcrow Membership list can be found in the Acknowledgments section.

% Use the asterisk to denote corresponding authorship and provide email address in note below.
%* correspondingauthor@institute.edu

% Please keep the abstract below 300 words
% Use "Eq" instead of "Equation" for equation citations.

% Results and Discussion can be combined.

\end{flushleft}

\section*{Abstract}


Comparative risk assessments to quantify the health potential
of active transportation, e.g., walking and cycling, have been
performed in the United States on a limited basis. Current
methodologies, which rely on extensive data collection and user
expertise, have estimated health benefits for select metropolitan
areas under a few alternative scenarios. These analyses have largely
been conducted using the Integrated Transport Health Impacts Model
(ITHIM) which was developed in the UK and has been adapted for use in
the US, primarily California. Here we present an implementation of the
active transportation component of ITHIM to model the health impacts
of modified active transportation levels. Using the newly developed
software package \package{} written in R, along with publicly
available national datasets, we quantify the change in disability
adjusted life years over a wide range of scenarios.  The R package
implemention of ITHIM's physical activity component provides a robust
environment for exploring the relationship between active
transportation and positive health outcomes.  A web-based
user-interface is available for users who want a first estimate for
the impact active transportation has on their
state/region.

%Comparative risk assessments to quantify the health potential
of active transportation, e.g., walking and cycling, have been
performed in the United States on a limited basis. Current
methodologies, which rely on extensive data collection and user
expertise, have estimated health benefits for select metropolitan
areas under a few alternative scenarios. These analyses have largely
been conducted using the Integrated Transport Health Impacts Model
(ITHIM) which was developed in the UK and has been adapted for use in
the US, primarily California. Here we present an implementation of the
active transportation component of ITHIM to model the health impacts
of modified active transportation levels. Using the newly developed
software package \package{} written in R, along with publicly
available national datasets, we quantify the change in disability
adjusted life years over a wide range of scenarios.  The R package
implemention of ITHIM's physical activity component provides a robust
environment for exploring the relationship between active
transportation and positive health outcomes.  A web-based
user-interface is available for users who want a first estimate for
the impact active transportation has on their
state/region.


% Please keep the Author Summary between 150 and 200 words
% Use first person. PLOS ONE authors please skip this step.
% Author Summary not valid for PLOS ONE submissions.
% \section*{Author Summary}
% \input{lib/authorSummary}

\linenumbers

\section*{Introduction}

Regular exercise has been shown to provide health benefits in terms of
chronic disease prevention \cite{warburton2006}, cognitive function
\cite{hillman2008}, and overall well being and satisfaction with life
\cite{maher2013}. Still only half of Americans obtain the recommended
levels of aerobic activity of 2.5 hours per week. There is
considerable variability in the attainment of this goal by location
within the country. Supportive local built and policy environments are
important determinants of population-level activity
patterns. Particularly, as they facilitate and foster routine and
moderate physical activity, such as walking, local decisions on design
and policy can play a huge role in improving the nation's health.

Given the constraints and complexity of local governance and
infrastructure investments, land use, and human behavior there is a
need for high quality and reliable data and tools for decision-making
related to urban transportaiton systems. The lack of accurate and
availbale estimates of the health impacts of such decisions has
limited the effectiveness of efforts to encourage non-motorized
transport in the US and in other countries. In the last decade new
tools, applying comparative risk assessment methodologies, have made
progress toward quantifying the multple health impacts of changes to
the transportation system and behaviors.

ITHIM is one such statistical model that integrates data on active
transport, physical activity, fine particulate matter and greenhouse
gas emissions to provide estimates for the proportional change in
mortality and morbidity for given baseline and alternate travel
scenarios. The model has been used to calculate the health impacts of
walking and bicycling short distances usually traveled by car or
driving low-emission automobiles \cite{woodcock2013,maizlish2013}.

ITHIM uses a comparative risk assessment framework for the active
transport component of the model.  We improve on the existing
implementation of the active transport component by including the
distribution of non-travel-related activity, improving numerical
precision when computing the population attributable fraction and
creating a simple user-interface for the custom-built \R{} package
\package{}.  We present this implementation here by using it to
estimate the relationship between overall, nationwide disease burden
and active transportation time in the United States.

\section*{Materials and Methods}


As with the original ITHIM implementation we, employed a comparative
risk assessment (CRA) across scenarios defined by mean active
transportation time, i.e., combined walking and cycling.  To do so we
investigate the proportional change in national disease burden between
the baseline model and a given alternative scenario, i.e., the
population attributable fraction, $\af$.  The population attributable
fraction is defined below and computed for each disease (ICD-10)
included in the model, namely Breast Cancer (C50), Colon Cancer
(C18-C21), Dementia (G30), Diabetes (E10-E14) and Cardiovascular
disease (I10, I12, I15, I20-I25, I30-I31, I33, I40).  We also include
the disability impacts of depression, though it is not listed as a
cause of death in CDC's data. For readability we omit an index
indicating disease.

\begin{equation}
\af_j = \frac{\int \! R_j(x)P(x) \, \mathrm{d}x  - \int \! R_j(x)Q(x) \,
  \mathrm{d}x}{\int \! R_j(x)P(x) \, \mathrm{d}x} \approx 1 - \frac{{\displaystyle \sum_{i=1}^n} R_j(x_i^\prime)}{{\displaystyle \sum_{i=1}^n} R_j(x_i)}  = 1 - \delta_j % = 1 - \frac{{\displaystyle \sum_{i=1}^n} e^{-\alpha \sqrt{x_i^\prime}}}{{\displaystyle \sum_{i=1}^n} e^{-\alpha \sqrt{x_i}}}
  \label{paf}
\end{equation}

Here $R(x)$ represents the risk when the individual has exposure $x$,
physical activity (\mets).  $P(x)$ and $Q(x)$ denote the population
density of individuals with exposure $x$, in the baseline and
alternate scenarios, respectively.  The exposure variable, $x$, is
modeled as the sum of two independent random variables, travel-related
and non-travel-related physical activity.  $\mathbf{x}$ and
$\mathbf{x}^\prime$ represent the quantiles for the exposure
distribution in baseline and alternative scenarios, respectively.
$\delta_j$ is the proportional burden for disease $j$ in the alternate
scenario.

\paragraph{Non-Travel-Related Physical Activity}

\begin{equation}
X_{ij}^{\mathrm{non-travel}} \sim \logNormalMath\left(\mu_{ij}, \gamma \mu_{ij}\right)
\end{equation}

Within each age-sex class the non-travel-related exposure is modeled
with a \logNormal{} distribution.  The age-sex specific mean ratios,
$r_{ij} = \frac{\mu_{ij}}{\mu_0}$ and $\gamma$ are estimated from the
American Time Use Survey data \cite{ATUS}.


\paragraph{Travel-Related Physical Activity}

The distribution for travel-related exposure is estimated from the
time spent walking or cycling, i.e., active transport time, which as
in the original ITHIM model is assigned a \logNormal{} distribution
with constant coefficient of variation across age-sex classes.
Travel-related exposure is estimated from active transport time using
assumptions about how much physical activity is required for cycling
and walking.



%To estimate the disease burden we first approximate the population
%attributable fraction through a discretization of the integral in
%equation \ref{paf}.

\paragraph{Non-Travel-Related Physical Activity}

With a simulated distribution for total physical activity in hand we
compute the empirical percentiles in the baseline and alternate
scenarios, $\mathbf{x}$ and $\mathbf{x}^\prime$. The proportional
change in disease burden, $\delta$, may be approximated as follows,
where $R$ is the risk function for total activity.

%% \begin{equation}
%% \delta = 1 - \af \approx \frac{{\displaystyle \sum_{i=1}^{n}} R(x_i)}{{\displaystyle \sum_{i=1}^{n}} R(x_i^\prime)}
%% \end{equation}


\paragraph{Estimation of Disease Burden}

The disease burden, in this case DALYs, is found for each of the
quantiles using equation \ref{burden} using $B_0$, the overall
nation-wide burden for the baseline.  The overall burden is then
computed as the sum across quintiles, $B=\sum_{j=1}^n B_j$.

\begin{equation}
B_j = \delta B_0\frac{\RR_j}{\sum_{k=1}^n \RR_j}\label{burden}
\end{equation}

\paragraph{Risk Functions}

We assign risk based on dose-response curves for each of the diseases
included in the model.

\begin{equation}
\RR_j(x) = \mathrm{e}^{-\alpha_j \tilde{x}}
\end{equation}

where $\tilde{x} = x^k$, $k=\frac{1}{2}$ and $r$ varies by age-sex
class and disease.  For a table of $r$ values see supplementary
material.

\paragraph{Nationwide Health Benefits to Active Transport}

For the purposes of this implementation we limited our analyses to
metropolitan regions of the United States. To idenfity urban
populations we used the National Center for Health Statistics 2013
Urban-Rural Classification Scheme for Counties \cite{ingram2014}. We
eliminated counties in the US with the classifcation of
`nonmetropolitan' (noncore or micropolitan). We used this and similar
classification schemes as well as county and state identifiers in
other datsets (NHTS, ATUS, CDC WONDER) to ensure consistency in our
estimates for input parameters.

To test the new implementation of ITHIM's active transportation
component, we used the US national metro population estimates for
active travel time (walking and cycling) as the baseline. We obtained
active travel time input parameters for alternate scenarios using
states with large sample sizes. From these examples we were able to
approximate low and high active travel scenarios. Finally, we applied
these active travel times to the US metro population to perform the
comparative risk assessment against the baseline US numbers. We then
examined the sensitivity of disease burden estimates to input
parameters including the active travel time and non-travel activity.

\begin{figure}[t]
  \centerline{\includegraphics[width=\textwidth]{./figures/fig1}}
    \caption{}\label{fig1}
\end{figure}

\begin{figure}[t]
  \centerline{\includegraphics[width=\textwidth]{./figures/fig2.pdf}}
    \caption{}\label{fig2}
\end{figure}

\begin{sidewaysfigure}[t]
  \centerline{\includegraphics[width=\textwidth]{./figures/fig3.pdf}}
  \caption{Estimates for disease burden nationwide in terms of thousands of
    DALYs for each of the six diseases included in the model.
  }\label{fig3}
\end{sidewaysfigure}



%The ITHIM model uses the comparative risk assesment framework to
estimate the population attributable fraction, $\af$.  In our case the
$\af$ represents the proportional change in national disease burden if
age and sex-specific active transport time means were changed.

\begin{equation}
\af = \frac{\int \! R(x)P(x) \, \mathrm{d}x  - \int \! R(x)Q(x) \, \mathrm{d}x}{\int \! R(x)P(x) \, \mathrm{d}x}
\end{equation}

Here $R(x)$ represents the risk when the individual has exposure $x$,
physical activity (\mets).  $P(x)$ and $Q(x)$ denote the population
density of individuals with exposure $x$, in the baseline and
alternate scenarios, respectively.  The exposure variable, $x$, is
modeled as the sum of two independent random variables, travel-related
exposure and non-travel-related exposure.  Non-travel-related exposure
is modeled with a \logNormal{} distribution with the age-sex specific
mean ratios for non-travel-related exposure are estimated from the
ATUS \cite{ATUS}.  The referent group mean is treated as a variable
and the coefficient of variation is treated as constant across age-sex
strata and estimated with ATUS.  The distribution for travel-related
exposure is estimated from the time spent walking or cycling, i.e.,
active transport time, which as in the original ITHIM model has
a \logNormal{} distribution with constant coefficient of variation
across age-sex strata.  Travel-related exposure is estimated from
active transport time using assumptions about how much physical
activity is required for cycling and walking.

%does this involve using the CDC Wonder data? we could add a bit here about it, but if this is just national estimtes and shifts perhaps we don't need it. 

For the purposes of this implementation we limited our analyses to metropolitan regions of the United States. To idenfity urban populations we used the National Center for Health Statistics 2013 Urban-Rural Classification Scheme for Counties \cite{ingram2014}. We eliminated counties in the US with the classifcation of 'nonmetropolitan' (noncore or micropolitan). We used this and similar classification schemes as well as county and state identifiers in other datsets (NHTS, ATUS, CDC WONDER) to ensure consistency in our estimates for input parameters. 


\section*{Results}


We report total change (averted or increased) in diability adjusted
life years (DALYs) as an estimate of disease burden change with
alternative scenarios.

To estimate DALYs we use a methodology adopted from previous US
implementations of ITHIM which scale Global Budern of Disease
Estimates for the US to smaller populations using mortality rate
ratios \cite{maizlish2013}. Age, sex, and casuse specific mortalities
were obtained form the CDC WONDER database for US metro counties for
the years 2010-2014. Cells with mortality counts less than 10 deaths
were suppressed for privacy and imputed (0 to 9) using R's random
integer generator.

Our subset of the US population to only metropolitan counties
consisted of 1,090 of 3,141 (34.7\%) counties, or approximately 272M
of 312M (87\%) of the total US population. The age and gender
distribution of our subset population was slightly younger (???making
this up???) than the national population. Baseline active transport
times (walking and cycling) for the US across age and sex groupings
was found to be approximately 43 min/week (???). Baseline non-travel
activity time was found to be ???X??? (min/week) from the ATUS for US
metropolitan populations.

Adapting ITHIM's methods to the R platform enables the rapid and
thorough assessment of parameter influence within the model. Another
important improvement enabled by the shift to a more computationaly
robust platform was to examine methodological assumptions of former
implementations. One critical assumption invovles the approximation of
??? used to calculate the attirbutable fraction (see Figure
\ref{fig:overview}). Previous implementations used quitiles to
approximate this value, essentailly the area under the ???? curve. We
improved upon those methods to use percentiles or 100 points to
estimate the same values. We find our estimates to be much more stable
and reliable at all values of active travel and non-travel activity
parameters.

As exhibited in Figure \ref{fig1}, the variability in disease burden
benefits varies to a much greater degree when using fewer points in
the attirbutable fraction approximation. Importantly, we see greater
stability when using the package and percentiles to approximate
attirbutable fraction.

Comparing baseline to scenarios with increased active travel produced
estimates of up to 360,000??? fewer DALYs (with 300\% increase in mean
active travel time) for our US population subset. We then varied the
mean non-travel activity time and active travel time to observe how
this chaged results (\ref{fig1}). We observe the most dramatic
responses in disease burden estimates to changes in active travel time
when non-travel activity times are lowest. This result is consistent
with the non-linear dose-response curve for physical activity and
several health outcomes.

%Our subset of the US population to only metropolitan counties
consisted of 1,090 of 3,141 (34.7\%) counties, or ???X??? of ???Y???
(???Z???\%) of the total US population. The age and gender
distribution of our subset population was slightly younger (???making
this up???) than the national population. Active transport times
(walking and cycling) for baseline, low, and high scenarios are shown
in \ref{fig2}. Overall mean walking time for high and low
scenarios were ???X??? (min/week) higher and ???Y??? (min/week) lower
than baseline, respectively. Baseline non-travel activity time was
estimated to be ???X??? (min/week) from the ATUS for US metropolitan
populations.

Comparing baseline to low/high scenarios produced estimates of ???X???
change in the disease burden (1,000s DALYs) for our US population
subset. We then varied the mean non-travel activity time and active
travel time to observe how this chaged results (\ref{fig1}). We
observe the most dramatic responses in disease burden estimates to
changes in active travel time when non-travel activity times are
lowest. This result is consistent with the non-linear dose-response
curve for physical activity and several health outcomes. The total
health benefit of increased active travel time (high scenario) varies
from ???X??? to ???Y??? with changing mean non-travel activity time (0
to 25 minutes).

Adapting ITHIM's methods to the R platform enables the rapid and
thorough assessment of parameter influence within the model. Another
important improvement enabled by the shift to a more computationaly
robust platform was to examine methodological assumptions of former
implementations. One critical assumption invovles the approximation of
??? used to calculate the attirbutable fraction (see
Figure \ref{fig:overview}). Previous implementations used quitiles to
approximate this value, essentailly the area under the ???? curve. We
improved upon those methods to use percentiles or 100 points to
estimate the same values. We find our estimates to be much more stable
and reliable at all values of active travel and non-travel activity
parameters.

As exhibited in Figure \ref{fig1}, the variability in disease burden
benfits varies to a much greater degree when using fewer points in the
attirbutable fraction approximation.


\section*{Discussion}
%\input{lib/discussion}

\section*{Conclusion}
%\input{lib/conclusion}

%\section*{Supporting Information}

%\input{lib/supportingInformation}

\section*{Acknowledgments}

\nolinenumbers

\bibliography{../tex/ITHIM}

\end{document}


% Template for PLoS
% Version 3.3 June 2016
%
% % % % % % % % % % % % % % % % % % % % % %
%
% -- IMPORTANT NOTE
%
% This template contains comments intended
% to minimize problems and delays during our production
% process. Please follow the template instructions
% whenever possible.
%
% % % % % % % % % % % % % % % % % % % % % % %
%
% Once your paper is accepted for publication,
% PLEASE REMOVE ALL TRACKED CHANGES in this file
% and leave only the final text of your manuscript.
% PLOS recommends the use of latexdiff to track changes during review, as this will help to maintain a clean tex file.
% Visit https://www.ctan.org/pkg/latexdiff?lang=en for info or contact us at latex@plos.org.
%
%
% There are no restrictions on package use within the LaTeX files except that
% no packages listed in the template may be deleted.
%
% Please do not include colors or graphics in the text.
%
% The manuscript LaTeX source should be contained within a single file (do not use \input, \externaldocument, or similar commands).
%
% % % % % % % % % % % % % % % % % % % % % % %
%
% -- FIGURES AND TABLES
%
% Please include tables/figure captions directly after the paragraph where they are first cited in the text.
%
% DO NOT INCLUDE GRAPHICS IN YOUR MANUSCRIPT
% - Figures should be uploaded separately from your manuscript file.
% - Figures generated using LaTeX should be extracted and removed from the PDF before submission.
% - Figures containing multiple panels/subfigures must be combined into one image file before submission.
% For figure citations, please use "Fig" instead of "Figure".
% See http://journals.plos.org/plosone/s/figures for PLOS figure guidelines.
%
% Tables should be cell-based and may not contain:
% - spacing/line breaks within cells to alter layout or alignment
% - do not nest tabular environments (no tabular environments within tabular environments)
% - no graphics or colored text (cell background color/shading OK)
% See http://journals.plos.org/plosone/s/tables for table guidelines.
%
% For tables that exceed the width of the text column, use the adjustwidth environment as illustrated in the example table in text below.
%
% % % % % % % % % % % % % % % % % % % % % % % %
%
% -- EQUATIONS, MATH SYMBOLS, SUBSCRIPTS, AND SUPERSCRIPTS
%
% IMPORTANT
% Below are a few tips to help format your equations and other special characters according to our specifications. For more tips to help reduce the possibility of formatting errors during conversion, please see our LaTeX guidelines at http://journals.plos.org/plosone/s/latex
%
% For inline equations, please be sure to include all portions of an equation in the math environment.  For example, x$^2$ is incorrect; this should be formatted as $x^2$ (or $\mathrm{x}^2$ if the romanized font is desired).
%
% Do not include text that is not math in the math environment. For example, CO2 should be written as CO\textsubscript{2} instead of CO$_2$.
%
% Please add line breaks to long display equations when possible in order to fit size of the column.
%
% For inline equations, please do not include punctuation (commas, etc) within the math environment unless this is part of the equation.
%
% When adding superscript or subscripts outside of brackets/braces, please group using {}.  For example, change "[U(D,E,\gamma)]^2" to "{[U(D,E,\gamma)]}^2".
%
% Do not use \cal for caligraphic font.  Instead, use \mathcal{}
%
% % % % % % % % % % % % % % % % % % % % % % % %
%
% Please contact latex@plos.org with any questions.
%
% % % % % % % % % % % % % % % % % % % % % % % %

% \subsection*{Etiam eget sapien nibh.}

% % For figure citations, please use "Fig" instead of "Figure".
% Nulla mi mi, Fig~\ref{fig1} venenatis sed ipsum varius, volutpat euismod diam. Proin rutrum vel massa non gravida. Quisque tempor sem et dignissim rutrum. Lorem ipsum dolor sit amet, consectetur adipiscing elit. Morbi at justo vitae nulla elementum commodo eu id massa. In vitae diam ac augue semper tincidunt eu ut eros. Fusce fringilla erat porttitor lectus cursus, \nameref{S1_Video} vel sagittis arcu lobortis. Aliquam in enim semper, aliquam massa id, cursus neque. Praesent faucibus semper libero.

% % Place figure captions after the first paragraph in which they are cited.
% \begin{figure}[!h]
% \caption{{\bf Bold the figure title.}
% Figure caption text here, please use this space for the figure panel descriptions instead of using subfigure commands. A: Lorem ipsum dolor sit amet. B: Consectetur adipiscing elit.}
% \label{fig1}
% \end{figure}

% Either type in your references using
% \begin{thebibliography}{}
% \bibitem{}
% Text
% \end{thebibliography}
%
% or
%
% Compile your BiBTeX database using our plos2015.bst
% style file and paste the contents of your .bbl file
% here.
%
% \begin{thebibliography}{10}

% % \bibitem{bib1}
% % Conant GC, Wolfe KH.
% % \newblock {{T}urning a hobby into a job: how duplicated genes find new
% %   functions}.
% % \newblock Nat Rev Genet. 2008 Dec;9(12):938--950.

% % \bibitem{bib2}
% % Ohno S.
% % \newblock Evolution by gene duplication.
% % \newblock London: George Alien \& Unwin Ltd. Berlin, Heidelberg and New York:
% %   Springer-Verlag.; 1970.

% % \bibitem{bib3}
% % Magwire MM, Bayer F, Webster CL, Cao C, Jiggins FM.
% % \newblock {{S}uccessive increases in the resistance of {D}rosophila to viral
% %   infection through a transposon insertion followed by a {D}uplication}.
% % \newblock PLoS Genet. 2011 Oct;7(10):e1002337.
% \input{../tex/ITHIM}
% \end{thebibliography}

%% Manuscript Organization

%% Manuscripts should be organized as follows. Instructions for each element appear below the list.
%% Beginning section

%% The following elements are required, in order:

%% Title page: List title, authors, and affiliations as first page of manuscript
%% Abstract
%% Introduction

%% Middle section

%% The following elements can be renamed as needed and presented in any order:

%% Materials and Methods
%% Results
%% Discussion
%% Conclusions (optional)

%% Ending section

%% The following elements are required, in order:

%% Acknowledgments
%% References
%% Supporting information captions (if applicable)

%% Other elements

%% Figure captions are inserted immediately after the first paragraph in which the figure is cited. Figure files are uploaded separately.
%% Tables are inserted immediately after the first paragraph in which they are cited.
%% Supporting information files are uploaded separately.
