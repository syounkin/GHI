Comparative risk assessments to quantify the health potential
of active transportation, e.g., walking and cycling, have been
performed in the United States on a limited basis. Current
methodologies, which rely on extensive data collection and user
expertise, have estimated health benefits for select metropolitan
areas under a few alternative scenarios. These analyses have largely
been conducted using the Integrated Transport Health Impacts Model
(ITHIM) which was developed in the UK and has been adapted for use in
the US, primarily California. Here we present an implementation of the
active transportation component of ITHIM to model the health impacts
of modified active transportation levels. Using the newly developed
software package \package{} written in R, along with publicly
available national datasets, we quantify the change in disability
adjusted life years over a wide range of scenarios.  The R package
implemention of ITHIM's physical activity component provides a robust
environment for exploring the relationship between active
transportation and positive health outcomes.  A web-based
user-interface is available for users who want a first estimate for
the impact active transportation has on their
state/region.
