As with the original ITHIM implementation we employed a comparative
risk analysis (CRA) across scenarios defined by mean active
transportation time, i.e., walking and cycling.

We compute the proportional change in national disease burden between
the baseline model and alternative scenarios, or population
attributable fraction, $\af$.  The population attributable fraction is
defined below, and computed for each disease included in the model,
namely Breast Cancer, Colon Cancer, Depression, Dementia, Diabetes and
Cardiovascular disease.

\begin{equation}
\af = \frac{\int \! R(x)P(x) \, \mathrm{d}x  - \int \! R(x)Q(x) \, \mathrm{d}x}{\int \! R(x)P(x) \, \mathrm{d}x}
\end{equation}

Here $R(x)$ represents the risk when the individual has exposure $x$,
physical activity (\mets).  $P(x)$ and $Q(x)$ denote the population
density of individuals with exposure $x$, in the baseline and
alternate scenarios, respectively.  The exposure variable, $x$, is
modeled as the sum of two independent random variables, travel-related
exposure and non-travel-related exposure.  Non-travel-related exposure
is modeled with a \logNormal{} distribution with the age-sex specific
mean ratios for non-travel-related exposure are estimated from the
ATUS \cite{ATUS}.  The referent group mean is treated as a variable
and the coefficient of variation is treated as constant across age-sex
strata and estimated with ATUS.  The distribution for travel-related
exposure is estimated from the time spent walking or cycling, i.e.,
active transport time, which as in the original ITHIM model has
a \logNormal{} distribution with constant coefficient of variation
across age-sex strata.  Travel-related exposure is estimated from
active transport time using assumptions about how much physical
activity is required for cycling and walking.

\paragraph{Disease Burden Estimation}
\begin{equation}
B_1 = B_0(1-\af)\frac{\RR_1}{\sum_j \RR_j}
\end{equation}

$B_j = \gamma_jB_1$, where $\gamma_i = \frac{\RR_i}{\RR_1}$.  The
overall burden is then computed as the sum across quintiles,
$B=\sum B_j$.  Finally, $B_0$ is estimated from CDC data\ldots


\begin{figure}[t]
  \centerline{\includegraphics[width=\textwidth]{./figures/fig1}}
    \caption{}\label{fig1}
\end{figure}

\begin{figure}[t]
  \centerline{\includegraphics[width=\textwidth]{./figures/fig2.pdf}}
    \caption{}\label{fig2}
\end{figure}

\begin{figure}[t]
  \centerline{\includegraphics[width=\textwidth]{./figures/fig3.pdf}}
  \caption{Estimates for disease burden nationwide in terms of thousands of
    DALYs for each of the six diseases included in the model.
  }\label{fig3}
\end{figure}

% does this involve using the CDC Wonder data? we could add a bit here
% about it, but if this is just national estimtes and shifts perhaps
% we don't need it.

For the purposes of this implementation we limited our analyses to
metropolitan regions of the United States. To idenfity urban
populations we used the National Center for Health Statistics 2013
Urban-Rural Classification Scheme for Counties \cite{ingram2014}. We
eliminated counties in the US with the classifcation of
'nonmetropolitan' (noncore or micropolitan). We used this and similar
classification schemes as well as county and state identifiers in
other datsets (NHTS, ATUS, CDC WONDER) to ensure consistency in our
estimates for input parameters.

To test the new implementation of ITHIM's active transportation
component, we used the UW national metro population estimates for
active travel time (walking and cycling) as the baseline. We obtained
active travel time input parameters for alternate scenarios using
states with large sample sizes. From these examples we were able to
approximate low and high active travel scenarios. Finally we applied
these active travel times to the US metro population to perform the
comparative risk assessment agains the baseline US numbers. We then
examined the sensitivity of disease burden estimates to input
parameters including the active travel time and non-travel activity.

We report total change (averted or increased) in diability adjusted
life years (DALYs) as an estimate of dissease burden change with
alternative scenarions. To estimate DALYs we use a methodology adopted
from previous US implementations of ITHIM which scale Global Budern of
Disease Estimates for the US to smaller populations using mortality
rate ratios \cite{maizlish2013}. Age, sex, and casuse specific
mortalities were obtained form the CDC WONDER database for US metro
counties for the years 2010-2014. Cells with less than 10 deaths were
imputed using R's random integer generator.
