As with the original ITHIM implementation we, employed a comparative
risk assessment (CRA) across scenarios defined by mean active
transportation time, i.e., combined walking and cycling.  To do so we
investigate the proportional change in national disease burden between
the baseline model and a given alternative scenario, i.e., the
population attributable fraction, $\af$.  The population attributable
fraction is defined below and computed for each disease (ICD-10) included in
the model, namely Breast Cancer (C50), Colon Cancer (C18-C21), Dementia (G30),
Diabetes (E10-E14) and Cardiovascular disease (I10, I12, I15, I20-I25, I30-I31, I33, I40).  We also include the disability impacts of depression, though it is not listed as a cause of death in CDC's data. For readability we omit an index
indicating disease.

\begin{equation}
\af = \frac{\int \! R(x)P(x) \, \mathrm{d}x  - \int \! R(x)Q(x) \,
  \mathrm{d}x}{\int \! R(x)P(x) \, \mathrm{d}x} \label{paf}
\end{equation}

Here $R(x)$ represents the risk when the individual has exposure $x$,
physical activity (\mets).  $P(x)$ and $Q(x)$ denote the population
density of individuals with exposure $x$, in the baseline and
alternate scenarios, respectively.  The exposure variable, $x$, is
modeled as the sum of two independent random variables, travel-related
and non-travel-related physical activity.

\paragraph{Non-Travel-Related Physical Activity}

\begin{equation}
X_{ij}^{\mathrm{non-travel}} \sim \logNormalMath\left(\mu_{ij}, \gamma \mu_{ij}\right)
\end{equation}

Within each age-sex class the non-travel-related exposure is modeled
with a \logNormal{} distribution.  The age-sex specific mean ratios,
$r_{ij} = \frac{\mu_{ij}}{\mu_0}$ and $\gamma$ are estimated from the
American Time Use Survey data \cite{ATUS}.

% latex table generated in R 3.3.1 by xtable 1.8-2 package
% Tue Nov  8 09:38:46 2016
\begin{table}[ht]
\centering
\begin{tabular}{rrr}
  \hline
 & M & F \\
  \hline
0-4 & 0.00 & 0.00 \\
  5-14 & 0.00 & 0.00 \\
  15-29 & 0.97 & 1.00 \\
  30-44 & 1.04 & 1.12 \\
  45-59 & 0.95 & 0.99 \\
  60-69 & 0.90 & 0.94 \\
  70-79 & 0.83 & 0.88 \\
  80+ & 0.72 & 0.77 \\
   \hline
\end{tabular}
\end{table}

\paragraph{Travel-Related Physical Activity}

The distribution for travel-related exposure is estimated from the
time spent walking or cycling, i.e., active transport time, which as
in the original ITHIM model is assigned a \logNormal{} distribution
with constant coefficient of variation across age-sex classes.
Travel-related exposure is estimated from active transport time using
assumptions about how much physical activity is required for cycling
and walking.

\paragraph{Estimation of Disease Burden}

To estimate the disease burden we first approximate the population
attributable fraction through a discretization of the integral in
equation \ref{paf}.


With a simulated distribution for total physical activity in hand we
compute the empirical percentiles in the baseline and alternate
scenarios, $\mathbf{x}$ and $\mathbf{x}^\prime$. The proportional
change in disease burden, $\delta$, may be approximated as follows,
where $R$ is the risk function for total activity.

\begin{equation}
\delta = 1 - \af \approx \frac{\sum_{i=1}^n R(x_i)}{\sum_{i=1}^n R(x_i^\prime)}
\end{equation}

The disease burden, in this case DALYs, is found for each of the
percentiles using equation \ref{burden}, using $B_0$, the overall nation-wide burden for
the baseline.  The overall burden is then computed as the sum across
quintiles, $B=\sum_{j=1}^n B_j$.

\begin{equation}
B_j = \delta B_0\frac{\RR_j}{\sum_{k=1}^n \RR_k}\label{burden}
\end{equation}

\paragraph{Risk Functions}

We assign risk based on dose-response curves for each of the diseases
included in the model.

\begin{equation}
\RR(x) = \mathrm{e}^{-r \tilde{x}}
\end{equation}

where $\tilde{x} = x^k$, $k=\frac{1}{2}$ and $r$ varies by age-sex
class and disease.  For a table of $r$ values see supplementary
material.

\paragraph{Nationwide Health Benefits to Active Transport}

For the purposes of this implementation we limited our analyses to
metropolitan regions of the United States. To idenfity urban
populations we used the National Center for Health Statistics 2013
Urban-Rural Classification Scheme for Counties \cite{ingram2014}. We
eliminated counties in the US with the classifcation of
`nonmetropolitan' (noncore or micropolitan). We used this and similar
classification schemes as well as county and state identifiers in
other datsets (NHTS, ATUS, CDC WONDER) to ensure consistency in our
estimates for input parameters.

To test the new implementation of ITHIM's active transportation
component, we used the US national metro population estimates for
active travel time (walking and cycling) as the baseline. We obtained
active travel time input parameters for alternate scenarios using
states with large sample sizes. From these examples we were able to
approximate low and high active travel scenarios. Finally, we applied
these active travel times to the US metro population to perform the
comparative risk assessment against the baseline US numbers. We then
examined the sensitivity of disease burden estimates to input
parameters including the active travel time and non-travel activity.

\begin{figure}[t]
  \centerline{\includegraphics[width=\textwidth]{./figures/fig1}}
    \caption{}\label{fig1}
\end{figure}

\begin{figure}[t]
  \centerline{\includegraphics[width=\textwidth]{./figures/fig2.pdf}}
    \caption{}\label{fig2}
\end{figure}

\begin{figure}[t]
  \centerline{\includegraphics[width=\textwidth]{./figures/fig3.pdf}}
  \caption{Estimates for disease burden nationwide in terms of thousands of
    DALYs for each of the six diseases included in the model.
  }\label{fig3}
\end{figure}
