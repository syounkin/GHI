We report total change (averted or increased) in diability adjusted
life years (DALYs) as an estimate of disease burden change with
alternative scenarios.

To estimate DALYs we use a methodology adopted from previous US
implementations of ITHIM which scale Global Budern of Disease
Estimates for the US to smaller populations using mortality rate
ratios \cite{maizlish2013}. Age, sex, and casuse specific mortalities
were obtained form the CDC WONDER database for US metro counties for
the years 2010-2014. Cells with mortality counts less than 10 deaths were suppressed for privacy and imputed (0 to 9) using
R's random integer generator.

Our subset of the US population to only metropolitan counties
consisted of 1,090 of 3,141 (34.7\%) counties, or approximately 272M of 312M
(87\%) of the total US population. The age and gender
distribution of our subset population was slightly younger (???making
this up???) than the national population. Baseline active transport times
(walking and cycling) for the US across age and sex groupings was found to be approximately 43 min/week (???). Baseline non-travel activity time was found to be ???X??? (min/week) from the ATUS for US metropolitan
populations.

Adapting ITHIM's methods to the R platform enables the rapid and
thorough assessment of parameter influence within the model. Another
important improvement enabled by the shift to a more computationaly
robust platform was to examine methodological assumptions of former
implementations. One critical assumption invovles the approximation of
??? used to calculate the attirbutable fraction (see
Figure \ref{fig:overview}). Previous implementations used quitiles to
approximate this value, essentailly the area under the ???? curve. We
improved upon those methods to use percentiles or 100 points to
estimate the same values. We find our estimates to be much more stable
and reliable at all values of active travel and non-travel activity
parameters.

As exhibited in Figure \ref{fig1}, the variability in disease burden
benefits varies to a much greater degree when using fewer points in the
attirbutable fraction approximation. Importantly, we see greater stability when using
the package and percentiles to approximate attirbutable fraction. 

Comparing baseline to scenarios with increased active travel produced estimates of up to
360,000??? fewer DALYs (with 300\% increase in mean active travel time) for our US population subset. We then varied the mean non-travel activity time and active
travel time to observe how this chaged results (\ref{fig1}). We
observe the most dramatic responses in disease burden estimates to
changes in active travel time when non-travel activity times are
lowest. This result is consistent with the non-linear dose-response
curve for physical activity and several health outcomes. 

